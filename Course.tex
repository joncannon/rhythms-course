\documentclass[11pt, reqno]{amsart}
\setlength{\columnsep}{20pt}
\usepackage{geometry}                % See geometry.pdf to learn the layout options. There are lots.
\geometry{letterpaper}                   % ... or a4paper or a5paper or ... 
%\geometry{landscape}                % Activate for for rotated page geometry
\usepackage[parfill]{parskip}    % Activate to begin paragraphs with an empty line rather than an indent
\usepackage{graphicx}
\usepackage{amssymb}
\usepackage{epstopdf}
\usepackage{amsthm}
\usepackage{lscape}
%\usepackage{multicol}
\usepackage{wrapfig}
\usepackage{grffile}
\usepackage{fancyhdr}
\usepackage{mdwlist}
\usepackage[T1]{fontenc}
\usepackage{empheq} 
\usepackage{chngcntr}
\usepackage{floatrow}
\DeclareGraphicsRule{.tif}{png}{.png}{`convert #1 `dirname #1`/`basename #1 .tif`.png}

\numberwithin{figure}{section}

\makeatletter
\renewcommand\section{\@startsection {section}{1}{\z@}%
                                   {-3.5ex \@plus -1ex \@minus -.2ex}%
                                   {2.3ex \@plus.2ex}%
                                   {\normalfont\Large\bfseries\scshape}}
\makeatother

   \renewcommand{\topfraction}{.8}
   \renewcommand{\bottomfraction}{.8}
   \renewcommand{\textfraction}{.2}


%\usepackage[style=numeric, maxnames=3,backend=bibtex, url=false, doi=false, isbn=false]{biblatex}
\bibliography{Paper1ac}
% Different font in captions
\newcommand{\captionfonts}{\small}

\newtheorem{thm}{Theorem}[subsection]
\newtheorem{lemma}{Lemma}[subsection]
\newtheorem{cor}{Corollary}[subsection]
\newtheorem{define}{Definition}[subsection]
\newtheorem{remark}{Remark}[subsection]

\makeatletter  % Allow the use of @ in command names
\long\def\@makecaption#1#2{%
  \vskip\abovecaptionskip
  \sbox\@tempboxa{{\captionfonts #1: #2}}%
  \ifdim \wd\@tempboxa >\hsize
    {\captionfonts #1: #2\par}
  \else
    \hbox to\hsize{\hfil\box\@tempboxa\hfil}%
  \fi
  \vskip\belowcaptionskip}
\makeatother   % Cancel the effect of \makeatletter

\setcounter{tocdepth}{2}
\renewcommand{\contentsname}{}

\counterwithout{figure}{section}
\counterwithout{figure}{subsection}


\numberwithin{equation}{section}

\title{Rhythms Course}
\author{Jon Cannon, Ben Polletta}
%\date{}                                           % Activate to display a given date or no date



\begin{document}
\maketitle
%\begin{multicols}{2}

%\tableofcontents

Content-goals:

\begin{enumerate}
\item Classical models of brain rhythms
\item Basic experimental/analysis techniques for rhythms
\end{enumerate}

Skill-goals:

\begin{enumerate}
\item Read experimental rhythms papers
\item Ask questions and hypothesize at the edge of the field
\item Turn experimental results into psuedo-computational models
\end{enumerate}

Assignments:
\begin{itemize}
\item Day 1: Papers accompanied by comprehension worksheets
\begin{itemize}
\item Restate
\item Interpret data/figures
\end{itemize}
\item Day 2: Synthesis assignment
\begin{itemize}
\item Extrapolate/hypothesize
\item Pose questions
\item Design model
\end{itemize}
\end{itemize}

Tools:
\begin{itemize}
\item Qualitative analysis of vector fields in phase plane
\end{itemize}


By week:
\begin{enumerate}

\item Introductory week
	\begin{itemize}
	\item Syllabus
	\item Review the neuron
		\begin{itemize}
		\item Neurons maintain electrical potential difference with environment
		\item Different types of ions are conducted in and out through channels, affecting potential
			\begin{itemize}
			\item Terminology: INWARD vs. OUTWARD, Polarizing vs. Hyperpolarizing
			\item Terminology: Conductance
			\end{itemize}
		\item Some ion channels open and close with voltage
		\item Neurons interface at synapses
			\begin{itemize}
			\item Some open INWARD excitatory channels
			\item Some open OUTWARD inhibitory channels.
			\end{itemize}
		\end{itemize}
	\item Introduce 2D vector fields
		\begin{itemize}
		\item The spike: V vs. n
		\item STATE determines DIRECTION
		\end{itemize}
	\item Introduce the frequency domain
		\begin{itemize}
		\item How to read a power spectrum
			\begin{itemize}
			\item What does a sine wave look like in the freq domain?
			\item What does a sum of sine waves look like?
			\item Look at other signal/spectrum pairs
			\end{itemize}
		\item How to read a spectrogram
		\end{itemize}
	\end{itemize}
\item Gamma 1: What does the data say?
	\begin{itemize}
	\item Fast-spiking cells are critical.
	\item Frequency is connected to inhibition time constant
	\end{itemize}
\item Gamma 2: What neural mechanism creates it?
\item Gamma 3: What function might it serve?

\textbf{PART 1: Cortico-thalamic rhythms}

\item Alpha 1: What does the data say?
\begin{itemize}
\item 
\end{itemize}
\item Alpha 2:  What neural mechanisms create it?
\begin{itemize}
\item 
\end{itemize}
\item Beta 1: What does the data say?
\begin{itemize}
\item 
\end{itemize}
\item Beta 2: What neural mechanisms create it?
\begin{itemize}
\item 
\end{itemize}

\textbf{PART 2: Hippocampal rhythms}

\item Theta 1: What does the data say?
\begin{itemize}
\item 
\end{itemize}
\item Theta 2: What neural mechanisms create it?
\begin{itemize}
\item 
\end{itemize}

\item Ripples 1: What does the data say?
\begin{itemize}
\item 
\end{itemize}
\item Ripples 2: What neural mechanisms create it?
\begin{itemize}
\item 
\end{itemize}

\item Ripples/Theta: What functions are theta and ripples serving
\begin{itemize}
\item 
\end{itemize}
\item Projects 1
\item Projects 2
\item Projects 3
\end{enumerate}


OLD OUTLINE


16 weeks

1a. Introduction: what are rhythms?
1b. Methods:

Gamma
Themes: Synchronous spiking, feedback inhibition
1.  
2.
3.

PART 1: Cortico-thalamic rhythms

Alpha
Themes: (Negative) attention, central control by thalamus, bursting
Papers:
Topics:
4. 
5.

Beta
Themes: Top-down control, laminar architecture
6.
7.

Delta
8.

PART 2: Cortico-hippocampal rhythms

Theta
Themes: Nested rhythms, phase coding
9.
10.
11.

SWR
Themes: Sequences, plasticity
12.
13.


14-16 Projects


\section{References}
%\printbibliography[heading=none]



\end{document}

